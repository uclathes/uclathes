% UCLA Policy on Vita
% For security reasons, UCLA policy specifies that your birth year and birth place should not be included in your vita.
%

\renewcommand{\vitastretch}{1.3}      % default is 1.67

\renewcommand{\vitadatewidth}{1.75in} % defaule is 1in

\renewcommand{\vitatextwidth}{4.5in}  % default is 5in
% Note \vitadatewidth + \vitatextwidth needs to be 6.25in since the table has  
% 3 column spacing. 3 spacing take 3 * 6 pt * 0.1384 in/pt = 0.2491 in.

\vitaitem   {1974 -- 1975}
                {Campus computer center ``User Services'' programmer and
                consultant, Stanford Center for Information Processing,
                Stanford University, Stanford, California.}
\vitaitem   {1975}
                {B.S.~(Mathematics) and A.B.~(Music),
                Stanford University.}
\vitaitem   {1977}
                {M.A.~(Music), UCLA, Los Angeles, California.}
\vitaitem   {1977 -- 1979}
                {Teaching Assistant, Computer Science Department, UCLA.
                Taught sections of Engineering 10 (beginning computer
                programming course) under direction of Professor Leon
                Levine.
                During summer 1979, taught a beginning programming
                course as part of the Freshman Summer Program.}
\vitaitem   {1979}
                {M.S.~(Computer Science), UCLA.}
\vitaitem   {1980 -- 1981 (expected)}
                {Research Assistant, Computer Science Department, UCLA.}
\vitaitem   {1981 -- present}
                {Programmer/Analyst, Computer Science Department, UCLA.}
                
%%%%%%%%%%%%%%%%%%%%%%%%%%%%%%%%%%%%%%%%%%%%%%%%%%%%%%%%%%%%%%%%%%%%%%%%
%                     Easy way to add publication
%%%%%%%%%%%%%%%%%%%%%%%%%%%%%%%%%%%%%%%%%%%%%%%%%%%%%%%%%%%%%%%%%%%%%%%%

%\publication    {Alexander Kusenko, Masahiro Kawasaki, Lauren Pearce, and Louis Yang, ``Relaxation leptogenesis, isocurvature perturbations, and the cosmic infrared background,'' (2017),\\ arXiv:1701.02175 [hep-ph].}

%\publication    {Alexander Kusenko, Lauren Pearce, and Louis Yang, ``Postinflationary Higgs relaxation and the origin of matter-antimatter asymmetry,'' Phys.~Rev.~Lett.~114, 061302 (2015), arXiv:1410.0722 [hep-ph].}

%%%%%%%%%%%%%%%%%%%%%%%%%%%%%%%%%%%%%%%%%%%%%%%%%%%%%%%%%%%%%%%%%%%%%%%%
%                         Customized CV content
%%%%%%%%%%%%%%%%%%%%%%%%%%%%%%%%%%%%%%%%%%%%%%%%%%%%%%%%%%%%%%%%%%%%%%%%
% Use \customCV{...} to type any words on the vita page in addition to the \vitaitem{} and \publication{} content.

\customCV{% Put anything you want to show here
    \vskip 30pt
%    \begin{center}
%        \header{Publications}%
%    \end{center}
%	\nobreak \vskip 12pt
    {\centering
        \header{Publications}\par%
    }
	\nobreak \vspace{6pt}
	\begin{enumerate}
	
	    \item Alexander Kusenko, Masahiro Kawasaki, Lauren Pearce, and Louis Yang, ``Relaxation leptogenesis, isocurvature perturbations, and the cosmic infrared background,'' \href{http://doi.org/10.1103/PhysRevLett.114.061302}{Phys.\ Rev.\ D \textbf{95}, 103006 (2017)}, \href{http://arxiv.org/abs/1701.02175}{arXiv:1701.02175 [hep-ph]}.
	
	    \item Alexander Kusenko, Lauren Pearce, and Louis Yang, ``Postinflationary Higgs relaxation and the origin of matter-antimatter asymmetry,'' \href{http://doi.org/10.1103/PhysRevD.95.103006}{Phys.\ Rev.\ Lett.\ \textbf{114}, 061302 (2015)}, \href{http://arxiv.org/abs/1410.0722}{arXiv:1410.0722 [hep-ph]}.
	
	\end{enumerate}
	
	\vskip 30pt
    {\centering
        \header{Presentations}\par%
    }
	\nobreak \vspace{6pt}
	\begin{enumerate}
	
	    \item ``Evolution of Scalar Fields in the Early Universe,'' at PACIFIC 2015 in Moorea, French Polynesia.
	
	    \item ``Post-inflationary Higgs relaxation and the origin of matter-antimatter asymmetry,'' at the 4th International Workshop on Dark Matter, Dark Energy and Matter-Antimatter Asymmetry in Hsinchu, Taiwan.
	
	\end{enumerate}
}
